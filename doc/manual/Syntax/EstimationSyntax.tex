\section{Estimation command and subcommand syntax\label{sec:estimation-syntax}}

\subsection{Random number seed}

\defComArg{Seed}{value}{Defines the random number generator seed}

\defArg{value}{The random number generator seed value}
\defType{Integer}
\defDefault{None}
\defValue{An integer between 0 and 10000 inclusive}

\subsection{Maximum posterior density (MPD)}

Two methods of minimising when estimating are, 
 
\begin{itemize}
	\item Numerical differences minimiser
  \item Differential evolution minimiser
\end{itemize}

Each type of minimiser requires a set of subcommands and arguments specific to that minimiser. Different minimisers can be called in sequence, with the results of one output as input to the next.

\defComLab{MPD}{Define the an estimater with label}

\defSub{Type} {Define the type of minimiser}
\defType{String}
\defDefault{NumericalDifferences}
\defValue{A valid type of minimiser, either NumericalDifferences or DESolver}

\subsubsection[Numerical differences minimiser]{\commandsubarg{MPD}{Type}{NumericalDifferences}} 

\defSub{MaxIterations} {Define the maximum number of iterations for the Numerical Differences minimiser}
\defType{Integer}
\defDefault{1000}
\defValue{A positive integer}

\defSub{MaxEvaluations} {Define the maximum number of evaluations for the Numerical Differences minimiser}
\defType{Integer}
\defDefault{4000}
\defValue{A positive integer}

\defSub{StepSize} {Define the step-size for the Numerical Differences minimiser}
\defType{Constant}
\defDefault{1e-6}
\defValue{A positive real number}

\defSub{GradientTolerance} {Define the gradient tolerance for the Numerical Differences minimiser}
\defType{Constant}
\defDefault{0.0002}
\defValue{A positive real number}

\subsubsection[Differential evolution minimiser]{\commandsubarg{MPD}{Type}{DESolver}} 

\defSub{MaxIterations} {Define the maximum number of iterations for the Differential Evolution minimiser}
\defType{Integer}
\defDefault{1000}
\defValue{A positive integer}

\defSub{MaxEvaluations} {Define the maximum number of evaluations for the Differential Evolution minimiser}
\defType{Integer}
\defDefault{4000}
\defValue{A positive integer}

\defSub{StepSize} {Define the step-size for the Differential Evolution minimiser}
\defType{Constant}
\defDefault{1e-6}
\defValue{A positive real number}

\defSub{GradientTolerance} {Define the gradient tolerance for the Differential Evolution minimiser}
\defType{Constant}
\defDefault{0.0002}
\defValue{A positive real number}

\subsection{Monte Carlo Markov Chain (MCMC)}

Only one method of carrying out MCMCs is available, Monte Carlo Markov Chain using Metropolis-Hastings

\defCom{MCMC}{Define the MCMC estimation arguments}

\defSub{Type} {Define the method of MCMC}
\defType{String}
\defDefault{MetropolisHastings}
\defValue{A valid type of MCMC, currently only Metropolis-Hastings is available}

\subsubsection[Metropolis-Hastings]{\commandsubarg{MCMC}{Type}{MetropolisHastings}} 

\defSub{Start} {Covariance multiplier for the starting point of the Markov chain}
\defType{Constant}
\defDefault{0}
\defValue{If 0, defines the starting point of the chain as the point estimate.	If >0, defines the starting point as randomly generated, with covariance matrix equal to the approximate covariance (inverse Hessian) times the value of this start parameter}

\defSub{Length} {Length of the Markov chain}
\defType{Integer}
\defDefault{None}
\defValue{Defines the length of the Markov chain (as a number of iterations)}

\defSub{keep} {Spacing between recorded values in the chain}
\defType{Integer}
\defDefault{1}
\defValue{Defines the spacing between recorded values in the chain. Samples from the posterior are written to file only if their sample number is evenly divisible by \argument{keep}}

\defSub{MaxCorrelation} {Maximum absolute correlation in the covariance matrix of the proposal distribution}
\defType{Constant}
\defDefault{0.8}
\defValue{Defines the maximum correlation in the covariance matrix of the proposal distribution. Correlations greater than MaxCorrelation are decreased to MaxCorrelation, and those less than -MaxCorrelation are increased to -MaxCorrelation }

\defSub{CorrelationAdjustmentMethod} {Method for adjusting small variances in the covariance proposal matrix}
\defType{String}
\defDefault{correlation}
\defValue{Defines the method (either correlation or covariance) for the adjusting small variances in the covariance matrix of the proposal distribution}

\defSub{CorrelationAdjustmentDiff} {Minimum nonzero variance times the range of the bounds in the covariance matrix of the proposal distribution}
\defType{Constant}
\defDefault{0.0001}
\defValue{Defines the minimum nonzero variance times the difference in the bounds of each parameter in the covariance matrix of the proposal distribution}

\defSub{StepSize} {Initial step-size (as a multiplier of the approximate covariance matrix)}
\defType{Constant}
\defDefault{$2.4d^{-0.5}$ where $d$ is the number of free parameters}
\defValue{The covariance of the proposal distribution is the approximate covariance (inverse Hessian) times this step-size parameter}

\defSub{ProporsalDistribution} {The shape of the proposal distribution (either \textit{t} or normal)}
\defType{String}
\defDefault{t}
\defValue{Either t or normal. Defines whether the proposal distribution should be multivariate t rather than multivariate normal}

\defSub{df} {Degrees of freedom of the multivariate t proposal distribution}
\defType{Integer}
\defDefault{4}
\defValue{Defines the degrees of freedom of the multivariate t proposal distribution}

\subsection{Profiles}

\defSub{Parameter} {Name of the parameter to be profiled}
\defType{String}
\defDefault{None}
\defValue{Defines the name of the parameter to be profiled}

\defSub{N} {Number of values at which to profile the parameter}
\defType{Integer}
\defDefault{10}
\defValue{Defines the number of values at which to profile the parameter}

\defSub{Lower} {lower bound on parameter}
\defType{Integer}
\defDefault{None}
\defValue{Defines the lower bound on the range of the parameter to profile}

\defSub{Upper} {Upper bound on parameter}
\defType{Integer}
\defDefault{None}
\defValue{Defines the upper bound on the range of the parameter to profile}

\subsection{Defining the free parameters and priors}

\defComArg{Estimate}{parameter\_name}{Define the name of the free parameter to estimate}
\defArg{parameter\_name}{The \SPM\ name of the parameter to estimate}
\defType{string}
\defDefault{None}
\defValue{A valid \SPM\ parameter name}

\defSub{Same}{Names of the other parameters which are constrained to have the same value}
\defType{String Vector}
\defDefault{None}
\defValue{Defines the names of all the other parameters which are constrained to have the same value as this parameter}

\defSub{Phase}{Phase at which this parameter should be estimated, in point estimation}
\defType{Integer}
\defDefault{1}
\defValue{Defines the phase at which this parameter should be freed}

\defSub{LowerBound}{Lower bound on this parameter}
\defType{Constant vector, of length equal to the parameter length}
\defDefault{None}
\defValue{Defines the lower bound(s) on this parameter}

\defSub{UpperBound}{Upper bound on this parameter}
\defType{Constant vector, of length equal to the parameter length}
\defDefault{None}
\defValue{Defines the upper bound(s) on this parameter}

\defSub{MCMCFixed}{Should this parameter be fixed during MCMC?}
\defType{Switch}
\defDefault{False}
\defValue{Define this parameter as fixed during MCMC (i.e., considered a constant for the MCMC)}

\defSub{Prior}{Defines the prior for this parameter}
\defType{String}
\defDefault{No default}
\defValue{Defines the type of prior on this parameter, and includes uniform, uniform-log, normal, normal-by-stdev, lognormal, Beta}

Different priors require different subcommands.

\subsubsection[Uniform prior]{\commandsubarg{estimate[\textit{label}]}{Prior}{Uniform}} 

The command \commandsubarg{estimate[\textit{label}]}{Prior}{Uniform} has no other subcommands.

\subsubsection[Uniform-log prior]{\commandsubarg{estimate[\textit{label}]}{Prior}{Uniform-Log}} 

The command \commandsubarg{estimate[\textit{label}]}{Prior}{Uniform-Log} has no other subcommands.

\subsubsection[Normal prior]{\commandsubarg{estimate[\textit{label}]}{Prior}{Normal}} 

\defSub{Mu}{Defines the mean $\mu$ of the normal prior}
\defType{Constant}
\defDefault{No default}
\defValue{Defines the mean of the normal prior}

\defSub{cv}{Defines the c.v. $c$ of the normal prior}
\defType{Constant}
\defDefault{No default}
\defValue{Defines the c.v. of the normal prior}

\subsubsection[Normal-by-stdev prior]{\commandsubarg{estimate[\textit{label}]}{Prior}{Normal-by-stdev}} 

\defSub{Mu}{Defines the mean $\mu$ of the normal by standard deviation prior}
\defType{Constant}
\defDefault{No default}
\defValue{Defines the mean of the normal by standard deviation prior}

\defSub{stdev}{Defines the standard deviation $\sigma$ of the normal by standard deviation prior}
\defType{Constant}
\defDefault{No default}
\defValue{Defines the standard deviation of the normal by standard deviation prior}

\subsubsection[Lognormal prior]{\commandsubarg{estimate[\textit{label}]}{Prior}{Lognormal}} 

\defSub{Mu}{Defines the mean $\mu$ of the lognormal prior}
\defType{Constant}
\defDefault{No default}
\defValue{Defines the mean of the lognormal prior}

\defSub{cv}{Defines the c.v. $c$ of the lognormal prior}
\defType{Constant}
\defDefault{No default}
\defValue{Defines the c.v. of the lognormal prior}

\subsubsection[Beta prior]{\commandsubarg{estimate[\textit{label}]}{Prior}{Beta}} 

\defSub{A}{The lower value of the range parameter $A$ of the Beta prior}
\defType{Constant}
\defDefault{No default}
\defValue{Defines the lower value of the range parameter $A$ of the Beta prior}

\defSub{B}{The upper value of the range parameter $B$ of the Beta prior}
\defType{Constant}
\defDefault{No default}
\defValue{Defines the upper value of the range parameter $B$ of the Beta prior}

\defSub{Mu}{Defines the mean $\mu$ of the Beta prior}
\defType{Constant}
\defDefault{No default}
\defValue{Defines the mean of the Beta prior}

\defSub{stdev}{Defines the standard deviation $\sigma$ of the Beta prior}
\defType{Constant}
\defDefault{No default}
\defValue{Defines the standard deviation of the Beta prior}
