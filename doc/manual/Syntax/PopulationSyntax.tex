\section{Population command and subcommand syntax\label{sec:population-syntax}}

\subsection{Model structure}

\defCom{Model}{Define the spatial structure, population structure, annual cycle, and model years}

\defSub{CellLength}{The length (distance) of one side of a cell}
\defType{Constant}
\defDefault{1}
\defValue{A positive real number}

\defSub{Nrows}{The number of rows $n_{rows}$ in the spatial structure}
\defType{Integer}
\defDefault{None}
\defValue{A positive integer, $n_{rows} > 0$}

\defSub{Ncols}{The number of columns $n_{cols}$ in the spatial structure}
\defType{Integer}
\defDefault{None}
\defValue{A positive integer, $n_{cols} > 0$}

\defSub{Layer}{The label for the base layer}
\defType{String}
\defDefault{None}
\defValue{Must be a label of a \argument{numeric} layer defined by \command{Layer}}

\defSub{Categories} {Labels of the categories (rows) of the population component of the partition}
\defType{Vector of strings, of length $1\ldots n_{categories}$}
\defDefault{None}
\defValue{Names of categories must be unique}

\defSub{MinAge}{Minimum age of the population}
\defType{Integer}
\defDefault{None}
\defValue{A non-negative integer, ${Age}_{min}\geq 0$ and ${Age}_{min}\leq {Age}_{max}$}

\defSub{MaxAge}{Maximum age of the population}
\defType{Integer}
\defDefault{None}
\defValue{A non-negative integer, ${Age}_{max}\geq 0$ and ${Age}_{min}\geq {Age}_{min}$}

\defSub{PlusGroup}{Define the largest age as a plus group}
\defType{Switch}
\defDefault{True}
\defValue{Defines  the largest age as a plus group}

\defSub{InitialisationPhases}{Define the labels of the phases of the initialisation}
\defType{Vector of strings, of length of the number of initialisation phases}
\defDefault{None}
\defValue{A valid label defined by \command{InitialisationPhases}}

\defSub{InitialYear}{Define the first year of the model, immediately following initialisation}
\defType{Integer}
\defDefault{None}
\defValue{Defines the first year of the model, $\geq 1$, e.g. 1990}

\defSub{CurrentYear}{Define the current year of the model}
\defType{Integer}
\defDefault{None}
\defValue{Defines the current year of the model, i.e., the model is run from \commandsub{Model}{.FirstYear}\ to \commandsub{Model}{.CurrentYear}}

\defSub{FinalYear}{Define the final year of the model in projections}
\defType{Integer}
\defDefault{None}
\defValue{Defines the final year of the model for use in projections, i.e., the model is run from \commandsub{Model}{.FirstYear} to \commandsub{Model}{.CurrentYear}, then projected to \commandsub{Model}{.FinalYear}}

\defSub{TimeSteps} {Define the \command{TimeStep} labels (in order that they are applied) to form the annual cycle}
\defType{String vector}
\defDefault{None}
\defValue{Defines the labels of the time steps that are run in each year}

\defComLab{InitialisationPhase}{Define the processes and years of the initialisation phase with label}

\defSub{Years} {Define the number of years to run}
\defType{Integer}
\defDefault{None}
\defValue{A non-negative integer}

\defSub{Processes} {Define the processes (in order of occurrence) to run in each year of the initialisation}
\defType{String vector}
\defDefault{None}
\defValue{A valid process label}

\defComLab{TimeStep} {Define a time step with label}

\defSub{Process} {Define the process labels, in the order that they are applied, for the time step}
\defType{String vector}
\defDefault{None}
\defValue{Defines the labels of the processes for that time step}

\subsection{Processes}

The population processes available are, 

\begin{itemize}
	\item Constant recruitment
  \item Beverton-Holt stock-recruit relationship recruitment
	\item Ageing
	\item Constant mortality rate
	\item Annually varying mortality rate
	\item Mortality event (as a number)
	\item Mortality event (as a biomass)
	\item Category transition
	\item Category shift
\end{itemize}

The movement processes available are,	

\begin{itemize}
	\item Preference movement
\end{itemize}

Each type of process requires a set of subcommands and arguments specific to that process.

\defComLab{Process} {Define a process with label}

\defSub{Type} {Define the type of process}
\defType{String}
\defDefault{None}
\defValue{A valid type of process}

\subsubsection[Constant recruitment]{\commandsubarg{Process}{Type}{ConstantRecruitment}} 

\defSub{R0} {Define the total amount of recruitment at equilibrium abundance levels}
\defType{Estimable}
\defDefault{None}
\defValue{Total amount (in numbers) of recruitment applied across all categories at equilibrium abundances}

\defSub{Categories} {Define the categories into which recruitment occurs}
\defType{String vector}
\defDefault{None}
\defValue{Valid categories from \commandsub{Model}{.Categories}}

\defSub{Proportions} {Define the proportion of recruitment that occurs into each category}
\defType{Estimable vector of length \commandsub{ConstantRecruitment}{[label].categories}}
\defDefault{None}
\defValue{Proportion of the annual recruitment that is applied to each category}

\defSub{Ages} {Define the ages within each category that receive recruitment}
\defType{Integer}
\defDefault{None}
\defValue{The age classes that receive recruitment}

\defSub{Layer} {Name of the layer used to determine where recruitment occurs}
\defType{String}
\defDefault{None}
\defValue{A valid layer as defined by \command{Layer}. If a numeric layer, then recruitment is in proportion to the layer values. If a logical layer, then recruitment occurs in cells where the layer value is \texttt{TRUE}}

\subsubsection[Beverton-Holt recruitment]{\commandsubarg{Process}{Type}{BHRecruitment}}

\defSub{R0} {Define the total amount of recruitment at equilibrium abundance levels}
\defType{Estimable}
\defDefault{None}
\defValue{Total amount (in numbers) of recruitment applied across all categories at equilibrium abundances}

\defSub{Categories} {Define the categories into which recruitment occurs}
\defType{String vector}
\defDefault{None}
\defValue{Valid categories from \commandsub{Model}{.Categories}}

\defSub{Proportions} {Define the proportion of recruitment that occurs into each category}
\defType{Estimable vector of length \commandsub{Process}{[label].Categories}}
\defDefault{None}
\defValue{Proportion of the annual recruitment that is applied to each category}

\defSub{Ages} {Define the age within each category that receive recruitment}
\defType{Integer}
\defDefault{None}
\defValue{The age classes that receive recruitment}

\defSub{Steepness} {Define the Beverton-Holt stock recruitment relationship steepness ($h$) parameter}
\defType{Estimable}
\defDefault{None}
\defValue{Steepness value between 0.2 and 1.0}

\defSub{SigmaR} {Define the recruitment variability $\sigma_R$ in the stock-recruitment relationship for projections}
\defType{Estimable}
\defDefault{None}

\defSub{Rho} {Define the autocorrelation $\rho$ in the recruitment variability in the stock-recruitment relationship for projections}
\defType{Estimable}
\defDefault{None}

\defSub{SSB} {Define the label of the \command{DerivedParameter} that defines the SSB}
\defType{String}
\defDefault{None}
\defValue{Must be a valid \command{DerivedParameter} label}

\defSub{YCS-Values} {YCS values}
\defType{Estimable vector}
\defDefault{None}
\defValue{Must be vector of length equal to \commandsub{BHRecruitmentProcess}{[label].YCS-Years}}
\defNote{Special values can be used here, i.e., mean, all}

\defSub{YCS-Years} {Years for YCS values}
\defType{Integer vector}
\defDefault{None}
\defValue{Must be vector of that specifies the years of \commandsub{BHRecruitmentProcess}{[label].YCS-Values}}
\defNote{Special year ranges (YYYY-YYYY) can be used}

\defSub{Layer} {Name of the layer used to determine where recruitment occurs}
\defType{String}
\defDefault{None}
\defValue{A valid layer as defined by \command{Layer}. If a numeric layer, then recruitment is in proportion to the layer values. If a logical layer, then recruitment occurs in cells where the layer value is \texttt{TRUE}}

\subsubsection[Ageing]{\commandsubarg{Process}{Type}{Ageing}}

\defSub{Categories} {Define the categories that ageing is applied to}
\defType{String vector}
\defDefault{None}
\defValue{Valid categories from \commandsub{Model}{.Categories}}

\subsubsection[Constant mortality rate]{\commandsubarg{Process}{Type}{ConstantMortalityRate}}

\defSub{M} {Define the constant mortality rate to be applied}
\defType{Estimable}
\defDefault{None}
\defValue{A real number $\ge 0$ and $\le 1$}

\defSub{Categories} {Define the categories that mortality is applied to}
\defType{String vector}
\defDefault{None}
\defValue{Valid categories from \commandsub{Model}{.Categories}}

\defSub{Selectivities} {Define the selectivities applied to each category}
\defType{String vector}
\defDefault{None}
\defValue{Valid selectivity labels defined by \command{Selectivity}}

\subsubsection[Annual mortality rate]{\commandsubarg{Process}{Type}{AnnualMortalityRate}}

\defSub{Years} {Define the years when the mortality rates are applied}
\defType{Constant vector}
\defDefault{None}
\defValue{Valid model years}

\defSub{M} {Define the mortality rate to be applied for each year}
\defType{Estimable vector}
\defDefault{None}
\defValue{A real number $\ge 0$ and $\le 1$}

\defSub{Categories} {Define the categories that mortality is applied to}
\defType{String vector}
\defDefault{None}
\defValue{Valid categories from \commandsub{Model}{.Categories}}

\defSub{Selectivities} {Define the selectivities applied to each category}
\defType{String vector}
\defDefault{None}
\defValue{Valid selectivity labels defined by \command{Selectivity}}

\subsubsection[Event mortality]{\commandsubarg{Process}{Type}{EventMortality}}

\defSub{Categories} {Define the categories that the event mortality is applied to}
\defType{String vector}
\defDefault{None}
\defValue{Valid categories from \commandsub{Model}{.Categories}}

\defSub{Years} {Define the years where the mortality even is applied}
\defType{Integer vector}
\defDefault{None}
\defValue{Valid years for the model}

\defSub{Layers} {Define the layers that specify the event mortality (as the abundance) in each year}
\defType{String vector, of length years}
\defDefault{None}
\defValue{Valid layers defined by \command{Layer}}

\defSub{Umax}{Define the maximum exploitation rate}
\defType{Constant}
\defDefault{None}
\defValue{Must be $\ge 0$ and $\le 1$}

\defSub{Selectivities} {Define the selectivities applied to each category}
\defType{String vector}
\defDefault{None}
\defValue{Valid selectivity labels defined by \command{Selectivity}}

\defSub{Penalty} {Define the event mortality penalty label}
\defType{String}
\defDefault{None}
\defValue{Valid penalty label defined by \command {Penalty}}

\subsubsection[Biomass event mortality]{\commandsubarg{Process}{Type}{BiomassEventMortality}}

\defSub{Categories}{Define the categories that the event mortality is applied to}
\defType{String vector}
\defDefault{None}
\defValue{Valid categories from \commandsub{Model}{.Categories}}

\defSub{Years}{Define the years where the mortality even is applied}
\defType{Integer vector}
\defDefault{None}
\defValue{Valid years for the model}

\defSub{Layers}{Define the layers that specify the event mortality (as a biomass) in each year}
\defType{String vector, of length years}
\defDefault{None}
\defValue{Valid layers defined by \command{Layer}}

\defSub{Umax} {Define the maximum exploitation rate}
\defType{Constant}
\defDefault{None}
\defValue{Must be $\ge 0$ and $\le 1$}

\defSub{Selectivities}{Define the selectivities applied to each category}
\defType{String vector}
\defDefault{None}
\defValue{Valid selectivity labels defined by \command{Selectivity}}

\defSub{Penalty} {Define the event mortality penalty label}
\defType{String}
\defDefault{None}
\defValue{Valid penalty label defined by \command {Penalty}}

\subsubsection[Category transition]{\commandsubarg{Process}{Type}{CategoryTransition}}

\defSub{From} {Define the category that is the source of the transition process}
\defType{String}
\defDefault{None}
\defValue{A valid category from \commandsub{Model}{.Categories}}

\defSub{To} {Define the category that is the sink of the transition process}
\defType{String}
\defDefault{None}
\defValue{A valid category from \commandsub{Model}{.Categories}}

\defSub{N} {Define the number of individuals to move}
\defType{Estimable}
\defDefault{None}
\defValue{A value $\ge 0$}

\defSub{Selectivity} {Define the selectivity applied to the source category}
\defType{String}
\defDefault{None}
\defValue{A valid selectivity label defined by \command{Selectivity}}

\subsubsection[Category transition rate]{\commandsubarg{Process}{Type}{CategoryTransitionRate}}

\defSub{From} {Define the category that is the source of the transition process}
\defType{String}
\defDefault{None}
\defValue{A valid category from \commandsub{Model}{.Categories}}

\defSub{To} {Define the category that is the sink of the transition process}
\defType{String}
\defDefault{None}
\defValue{A valid category from \commandsub{Model}{.Categories}}

\defSub{Proportion} {Define the proportion of individuals to move}
\defType{Estimable}
\defDefault{None}
\defValue{A value $\ge 0$ and $\le 1$}

\defSub{Selectivity} {Define the selectivity applied to the source category}
\defType{String}
\defDefault{None}
\defValue{A valid selectivity label defined by \command{Selectivity}}

\subsubsection[Preference movement]{\commandsubarg{Process}{Type}{PreferenceMovement}}

\defSub{Categories} {Define the categories that the preference function movement is applied to}
\defType{String vector}
\defDefault{None}
\defValue{Valid categories from \commandsub{Model}{.Categories}}

\defSub{PreferenceFunctions} {Define the labels of the individual  preference functions that make up the total preference function}
\defType{String vector}
\defDefault{None}
\defValue{Valid preference function labels defined by \command{PreferenceFunction}}

\subsection{Preference functions}

The individual preference functions available are, 

\begin{itemize}
	\item Constant
	\item Normal
	\item Double-normal
	\item Logistic
	\item Inverse logistic
	\item Exponential-decay
	\item Threshold
	\item Threshold-biomass	
\end{itemize}

Each type of preference function requires a set of subcommands and arguments specific to that function.

\defComLab{PreferenceFunction} {Define a preference function with label}

\defSub{Type} {Define the type of preference function}
\defType{String}
\defDefault{None}
\defValue{A valid type of preference function}

\subsubsection[Constant]{\commandsubarg{PreferenceFunction}{Type}{Constant}}

\defSub{Layer} {Defines the layer which supplies the preference function independent variable}
\defType{String}
\defDefault{None}
\defValue{A valid layer defined by \command{Layer}}

\defSub{Alpha} {Defines the multiplicative constant $\alpha$}
\defType{Estimable}
\defDefault{None}

\subsubsection[Normal]{\commandsubarg{PreferenceFunction}{Type}{Normal}}

\defSub{Layer} {Defines the layer which supplies the preference function independent variable}
\defType{String}
\defDefault{None}
\defValue{A valid layer defined by \command{Layer}}

\defSub{Alpha} {Defines the multiplicative constant $\alpha$}
\defType{Estimable}
\defDefault{None}

\defSub{Mu} {Defines the $\mu$ parameter of the normal preference function}
\defType{Estimable}
\defDefault{None}

\defSub{Sigma} {Defines the $\sigma$ parameter of the normal preference function}
\defType{Estimable}
\defDefault{None}

\subsubsection[Double-normal]{\commandsubarg{PreferenceFunction}{Type}{Double-normal}}

\defSub{Layer} {Defines the layer which supplies the preference function independent variable}
\defType{String}
\defDefault{None}
\defValue{A valid layer defined by \command{Layer}}

\defSub{Alpha} {Defines the multiplicative constant $\alpha$}
\defType{Estimable}
\defDefault{None}

\defSub{Mu} {Defines the $\mu$ parameter of the double-normal preference function}
\defType{Estimable}
\defDefault{None}

\defSub{SigmaL} {Defines the $\sigma_L$ parameter of the double-normal preference function}
\defType{Estimable}
\defDefault{None}

\defSub{SigmaR} {Defines the $\sigma_R$ parameter of the double-normal preference function}
\defType{Estimable}
\defDefault{None}

\subsubsection[Logistic]{\commandsubarg{PreferenceFunction}{Type}{Logistic}}

\defSub{Layer} {Defines the layer which supplies the preference function independent variable}
\defType{String}
\defDefault{None}
\defValue{A valid layer defined by \command{Layer}}

\defSub{Alpha} {Defines the multiplicative constant $\alpha$}
\defType{Estimable}
\defDefault{None}

\defSub{a50} {Defines the $a_{50}$ parameter of the logistic preference function}
\defType{Estimable}
\defDefault{None}

\defSub{ato95} {Defines the $a_{to95}$ parameter of the logistic preference function}
\defType{Estimable}
\defDefault{None}

\subsubsection[Inverse-logistic]{\commandsubarg{PreferenceFunction}{Type}{Inverse-logistic}}

\defSub{Layer} {Defines the layer which supplies the preference function independent variable}
\defType{String}
\defDefault{None}
\defValue{A valid layer defined by \command{Layer}}

\defSub{Alpha} {Defines the multiplicative constant $\alpha$}
\defType{Estimable}
\defDefault{None}

\defSub{a50} {Defines the $a_{50}$ parameter of the inverse-logistic preference function}
\defType{Estimable}
\defDefault{None}

\defSub{ato95} {Defines the $a_{to95}$ parameter of the inverse-logistic preference function}
\defType{Estimable}
\defDefault{None}

\subsubsection[Exponential-decay]{\commandsubarg{PreferenceFunction}{Type}{Exponential-decay}}

\defSub{Layer} {Defines the layer which supplies the preference function independent variable}
\defType{String}
\defDefault{None}
\defValue{A valid layer defined by \command{Layer}}

\defSub{Alpha} {Defines the multiplicative constant $\alpha$}
\defType{Estimable}
\defDefault{None}

\defSub{Lambda} {Defines the $\lambda$ parameter of the exponential-decay preference function}
\defType{Estimable}
\defDefault{None}

\subsubsection[Threshold]{\commandsubarg{PreferenceFunction}{Type}{Threshold}}

\defSub{Layer} {Defines the layer which supplies the preference function independent variable}
\defType{String}
\defDefault{None}
\defValue{A valid layer defined by \command{Layer}}

\defSub{Alpha} {Defines the multiplicative constant $\alpha$}
\defType{Estimable}
\defDefault{None}

\defSub{N} {Defines the $N$ parameter of the threshold preference function}
\defType{Estimable}
\defDefault{None}

\defSub{Lambda} {Defines the $\lambda$ parameter of the threshold preference function}
\defType{Estimable}
\defDefault{None}

\subsubsection[Threshold-biomass]{\commandsubarg{PreferenceFunction}{Type}{Threshold-biomass}}

\defSub{Layer} {Defines the layer which supplies the preference function independent variable}
\defType{String}
\defDefault{None}
\defValue{A valid layer defined by \command{Layer}}

\defSub{Alpha} {Defines the multiplicative constant $\alpha$}
\defType{Estimable}
\defDefault{None}

\defSub{Biomass} {Defines the $B$ biomass parameter of the threshold biomass preference function}
\defType{Estimable}
\defDefault{None}

\defSub{Lambda} {Defines the $\lambda$ parameter of the threshold biomass preference function}
\defType{Estimable}
\defDefault{None}

\subsection{Layers}

The available layer types  are, 

\begin{itemize}
	\item Categorical
	\item Numeric
	\item Meta-layer
\end{itemize}

\defComLab{Layer} {Define a layer function with label}

\defSub{Type} {Define the type of layer}
\defType{String}
\defDefault{None}
\defValue{A valid type of layer}

\subsubsection[Categorical]{\commandsubarg{Layer}{Type}{Categorical}}

\defSub{Row$x$} {Define the values of the layer for row $x$}
\defType{String vector, of length \commandsub{Model}{.ncols}}
\defDefault{None}
\defValue{A vector of values of length equal to the number of columns defined for the spatial structure. There must be exactly $x$ rows defined, one for each row of the spatial structure}

\subsubsection[Numeric]{\commandsubarg{Layer}{Type}{Numeric}}

\defSub{Row$x$} {Define the values of the layer for row $x$}
\defType{Constant vector, of length \commandsub{Model}{.ncols}}
\defDefault{None}
\defValue{A vector of values of length equal to the number of columns defined for the spatial structure. There must be exactly $x$ rows defined, one for each row of the spatial structure}

\defSub{Rescale} {Rescale values of the layer}
\defType{Constant}
\defDefault{1.0}
\defValue{Rescales the values in the layer so that they have maximum value defined by the rescaling constant}

\subsubsection[Meta-layer]{\commandsubarg{Layer}{Type}{Meta-layer}}

\defSub{Year} {Define the years}
\defType{Constant vector, with values for each year of the model}
\defDefault{None}

\defSub{Layers} {Define the layer labels for each of the years}
\defType{String vector, with values for each \argument{year} specified}
\defDefault{None}
\defCondition{Layers cannot be of type \argument{meta-layer}}

\subsection{Derived quantities}

The individual types of derived quantities available are, 

\begin{itemize}
	\item Abundance
	\item Biomass
\end{itemize}

\defComLab{DerivedQuantity} {Define a derived quantity with label}

\defSub{Type} {Define the type of derived quantity}
\defType{String}
\defDefault{None}
\defValue{A valid type of derived quantity}

\subsubsection[Abundance]{\commandsubarg{DerivedQuantity}{Type}{Abundance}}

\defSub{Categories} {Define the categories are used to calculate the derived quantity}
\defType{String vector}
\defDefault{None}
\defValue{Valid categories from \commandsub{Model}{.Categories}}

\defSub{Selectivity} {Define the selectivities}
\defType{String vector}
\defDefault{None}
\defValue{Valid selectivity labela from \command{Selectivity}}

\defSub{TimeStep} {Define the time step at the end of which, the derived quantity is calculated}
\defType{String}
\defDefault{None}
\defValue{A valid time step label from \command{TimeStep}}

\subsubsection[Biomass]{\commandsubarg{DerivedQuantity}{Type}{Biomass}}

\defSub{Categories} {Define the categories are used to calculate the derived quantity}
\defType{String vector}
\defDefault{None}
\defValue{Valid categories from \commandsub{Model}{.Categories}}

\defSub{Selectivity} {Define the selectivities}
\defType{String vector}
\defDefault{None}
\defValue{Valid selectivity labela from \command{Selectivity}}

\defSub{TimeStep} {Define the time step at the end of which, the derived quantity is calculated}
\defType{String}
\defDefault{None}
\defValue{A valid time step label from \command{TimeStep}}

\subsection{Size-at-age}

The individual types of size-at-age available are, 

\begin{itemize}
	\item von Bertalanffy
	\item Schnute
\end{itemize}

\defComLab{SizeAtAge} {Define a size-at-age relationship with label}

\defSub{Type} {Define the type of relationship}
\defType{String}
\defDefault{None}
\defValue{A valid type of size-at-age relationship}

\subsubsection[von Bertalanffy]{\commandsubarg{DerivedQuantity}{Type}{vonBert}}

\defSub{Linf} {Define the $L_\infty$ parameter of the von Bertalanffy relationship}
\defType{Estimable}
\defDefault{None}
\defValue{A positive real number}

\defSub{k} {Define the $k$ parameter of the von Bertalanffy relationship}
\defType{Estimable}
\defDefault{None}
\defValue{A positive real number}

\defSub{t0} {Define the $t_0$ parameter of the von Bertalanffy relationship}
\defType{Estimable}
\defDefault{None}
\defValue{A real number}

\defSub{GrowthProps} {Define the proportion of the year for each time step for evaluating size}
\defType{Constant vector}
\defDefault{None}
\defValue{A vector of values, $\le1$ of length equal to the number of time steps}

\subsubsection[Schnute]{\commandsubarg{DerivedQuantity}{Type}{Schnute}}

\subsection{Selectivities}
