\section{Reporting command and subcommand syntax\label{sec:report-syntax}}

\subsection{Reports}

The report types available are,

\begin{description}
  \item Print the partition
  \item Print a derived quantity
  \item Print the estimated parameters in a vector format (suitable for use with \texttt{spm -i})
  \item Print a summary of the estimated parameters
  \item Print the objective function values
  \item Print an observation values, fits, and residuals
  \item Print the covariance matrix
  \item Print a layer
  \item Print a selectivity's values
  \item Print the weight-at-size using the size-weight relationship
\end{description}

Each type of report requires a set of subcommands and arguments specific to that report.

\defComLab{report}{Define an output report}

\defSub{type} {Define the type of report}
\defType{String}
\defDefault{None}
\defValue{A valid type of report}

\subsubsection[Print the partition]{\commandlabsubarg{report}{type}{partition}}

\defSub{year} {Define the year that the partition report applies to}
\defType{Integer}
\defDefault{None}
\defValue{A positive integer between \commandsub{model}{initial\_year} and \commandsub{model}{current\_year}}

\defSub{time\_step} {Define the time-step that the partition report applies to}
\defType{Integer}
\defDefault{None}
\defValue{A valid time-step}

\defSub{file\_name} {Define the name of the output file where the report is written}
\defType{String}
\defDefault{None}
\defValue{A valid file name. If not supplied, then output is directed to the standard out}

\defSub{overwrite} {Specify if any previous file with the same name as the output file should be overwritten or appended to}
\defType{Switch}
\defDefault{True}
\defValue{Either True of False}

\subsubsection[Print the partition at initialisation]{\commandlabsubarg{report}{type}{initialisation}}

\defSub{phase} {Define the phase of initialisation that the partition report applies to}
\defType{string}
\defDefault{None}
\defValue{A valid phase label, from \command{initialisation\_phase}}

\defSub{file\_name} {Define the name of the output file where the report is written}
\defType{String}
\defDefault{None}
\defValue{A valid file name. If not supplied, then output is directed to the standard out}

\defSub{overwrite} {Specify if any previous file with the same name as the output file should be overwritten or appended to}
\defType{Switch}
\defDefault{True}
\defValue{Either True of False}

\subsubsection[Print a derived quantity]{\commandlabsubarg{report}{type}{derived\_quantity}}

\defSub{derived\_quantity} {Define the label of the derived quantity to print}
\defType{String}
\defDefault{None}
\defValue{A valid label from \command{derived\_quantity}}

\defSub{file\_name} {Define the name of the output file where the report is written}
\defType{String}
\defDefault{None}
\defValue{A valid file name. If not supplied, then output is directed to the standard out}

\defSub{overwrite} {Specify if any previous file with the same name as the output file should be overwritten or appended to}
\defType{Switch}
\defDefault{True}
\defValue{Either True of False}

\subsubsection[Printing the estimated parameter values out as a vector]{\commandlabsubarg{report}{type}{estimate\_values}\label{sec:InputFileFormat}}

Prints the estimated parameters in a format suitable for use with \texttt{spm -i}.

\defSub{file\_name} {Define the name of the output file where the report is written}
\defType{String}
\defDefault{None}
\defValue{A valid file name. If not supplied, then output is directed to the standard out}

\defSub{overwrite} {Specify if any previous file with the same name as the output file should be overwritten or appended to}
\defType{Switch}
\defDefault{True}
\defValue{Either True of False}

\subsubsection[Print a summary of the estimated parameters]{\commandlabsubarg{report}{type}{estimate\_summary}}

\defSub{file\_name} {Define the name of the output file where the report is written}
\defType{String}
\defDefault{None}
\defValue{A valid file name. If not supplied, then output is directed to the standard out}

\defSub{overwrite} {Specify if any previous file with the same name as the output file should be overwritten or appended to}
\defType{Switch}
\defDefault{True}
\defValue{Either True of False}

\subsubsection[Print the Objective function values]{\commandlabsubarg{report}{type}{objective\_function}}

\defSub{file\_name} {Define the name of the output file where the report is written}
\defType{String}
\defDefault{None}
\defValue{A valid file name. If not supplied, then output is directed to the standard out}

\defSub{overwrite} {Specify if any previous file with the same name as the output file should be overwritten or appended to}
\defType{Switch}
\defDefault{True}
\defValue{Either True of False}

\subsubsection[Print a a summary of the an observation, including fits, and residuals]{\commandlabsubarg{report}{type}{observation}}

\defSub{observation} {Define the label of the observation to print}
\defType{String}
\defDefault{None}
\defValue{A valid label from \command{Observation}}

\defSub{file\_name} {Define the name of the output file where the report is written}
\defType{String}
\defDefault{None}
\defValue{A valid file name. If not supplied, then output is directed to the standard out}

\defSub{overwrite} {Specify if any previous file with the same name as the output file should be overwritten or appended to}
\defType{Switch}
\defDefault{True}
\defValue{Either True of False}

\subsubsection[Print a layer]{\commandlabsubarg{report}{type}{layer}}

\defSub{layer} {Define the label of the layer to print}
\defType{String}
\defDefault{None}
\defValue{A valid label from \command{Layer}}

\defSub{year} {Define the year for the printing of the layer}
\defType{Integer}
\defDefault{None}
\defValue{A positive integer between \commandsub{model}{initial\_year} and \commandsub{model}{current\_year}}

\defSub{time\_step} {Define the time-step for the printing of the layer}
\defType{Integer}
\defDefault{None}
\defValue{A valid time-step}

\defSub{file\_name} {Define the name of the output file where the report is written}
\defType{String}
\defDefault{None}
\defValue{A valid file name. If not supplied, then output is directed to the standard out}

\defSub{overwrite} {Specify if any previous file with the same name as the output file should be overwritten or appended to}
\defType{Switch}
\defDefault{True}
\defValue{Either True of False}

\subsubsection[Print a selectivity]{\commandlabsubarg{report}{type}{selectivity}}

\defSub{selectivity} {Define the label of the selectivity to print}
\defType{String}
\defDefault{None}
\defValue{A valid label from \command{Selectivity}}

\defSub{year} {Define the year for the printing of the selectivity}
\defType{Integer}
\defDefault{None}
\defValue{A positive integer between \commandsub{model}{initial\_year} and \commandsub{model}{current\_year}}

\defSub{time\_step} {Define the time-step for the printing of the selectivity}
\defType{Integer}
\defDefault{None}
\defValue{A valid time-step}

\defSub{file\_name} {Define the name of the output file where the report is written}
\defType{String}
\defDefault{None}
\defValue{A valid file name. If not supplied, then output is directed to the standard out}

\defSub{overwrite} {Specify if any previous file with the same name as the output file should be overwritten or appended to}
\defType{Switch}
\defDefault{True}
\defValue{Either True of False}

\subsubsection[Print the random number seed used]{\commandlabsubarg{report}{type}{random\_number\_seed}}

\defSub{file\_name} {Define the name of the output file where the report is written}
\defType{String}
\defDefault{None}
\defValue{A valid file name. If not supplied, then output is directed to the standard out}

\defSub{overwrite} {Specify if any previous file with the same name as the output file should be overwritten or appended to}
\defType{Switch}
\defDefault{True}
\defValue{Either True of False}

\subsubsection[Print the weight-at-size]{\commandlabsubarg{report}{type}{weight\_at\_size}}

\defSub{size\_weight} {Define the label of the size-weight relationship print}
\defType{String}
\defDefault{None}
\defValue{A valid label from \command{size\_weight}}

\defSub{sizes} {Define the label of the size-weight relationship print}
\defType{Constant}
\defDefault{None}
\defValue{Values of sizes to calculate the size-weight relationship for}

\defSub{file\_name} {Define the name of the output file where the report is written}
\defType{String}
\defDefault{None}
\defValue{A valid file name. If not supplied, then output is directed to the standard out}

\defSub{overwrite} {Specify if any previous file with the same name as the output file should be overwritten or appended to}
\defType{Switch}
\defDefault{True}
\defValue{Either True of False}
