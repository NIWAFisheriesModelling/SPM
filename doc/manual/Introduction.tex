\section{Introduction\label{sec:Introduction}}

The \SPMName\ (\SPM) is a generalised spatially explicit age-structured population dynamics and movement model. \SPM\ can model population dynamics and movement parameters for an age-structured population using a range of observations, including tagging, relative abundance, and age frequency data. \SPM\ implements an age-structured population within an arbitrary shaped spatial structure, which can have user defined categories (e.g., immature, mature, male, female, etc.), and age range. Movement can be modelled as either adjacent cell movements or global movements based on covariates.

This manual describes how to use \SPM, including how to run \SPM, how to set up an \config. Further, we describe the population dynamics and estimation methods, and describe how to specify and interpret output. If you are new to \SPM, then a good place to start is by reading this manual and attempting to replicate the examples (Section \ref{sec:examples}).

\subsection{Version\label{sec:version}}

This document (last modified \DocVer) describes \SPM\ \VER. The \SPM\ version number is suffixed with a date/time (\texttt{yyyy-mm-dd}) and revision number, giving the revision control system UTC date and revision number for the most recent modification of the source files. User manual updates will usually be issued for each minor version or date release of \SPM, and can be obtained, on request, from the authors.\index{Version}

\subsection{Citing \SPM}

A suitable reference for \SPM\ and this document is:

\ManualRef\index{Citation}\index{Citing \SPM}

\subsection{\I{Software license}\index{Common Public License}}

This program and the accompanying materials are made available under the terms of the \href{http://www.opensource.org/licenses/cpl1.0.php}{Common Public License v1.0} which accompanies this distribution (Section \ref{sec:Common-Public-License}).

Copyright \copyright 2008-\SourceControlYearDoc, \href{http://www.niwa.co.nz}{\Organisation} and the \href{http://www.fish.govt.nz}{New Zealand Ministry of Fisheries}. All rights reserved.

\subsection{\I{System requirements}}

\SPM\ is available for most IBM compatible machines running \I{Linux} and from the command prompt under most \I{Microsoft Windows} operating systems.

Several of \SPM s tasks are highly computer intensive and a fast, powerful processor is recommended. We recommend a minimum of 10 megabytes of free RAM (although, depending on the scope of the problem, you may need much more). Some of \SPM s tasks can be multi-threaded, and hence multi-core machines may perform some tasks considerably quicker than single core processors. The program itself requires only a few megabytes of hard-disk space but output files can consume large amounts of disk space. Depending on number and type of user output requests, the output could range from a few hundred kilobytes to several hundred megabytes. However, we note that, depending on the model implemented, some of \SPM s tasks can take a considerable amount of time.

\subsection{Necessary files}

In Linux, only the executable file \texttt{spm} is required to run \SPM\ (but, depending on your system, you may need the either the 32- or 64-bit version). For Microsoft Windows, you need the executable file \texttt{spm.exe}. There is no 64-bit version for Microsoft Windows.

\subsection{Useful add-ons}

No software other than the appropriate operating system or emulation package is required to run \SPM. However, as \SPM\ offers little in the way of  post-processing of the output, most users will wish to have a package available that allows tabulation and graphing of model outputs. We recommend the use of software packages such as \href{http://www.microsoft.com}{Microsoft Excel}, \href{http://www.insightful.com}{S-Plus}, or \href{http://www.r-project.org}{\R}\ (R Development Core Team 2007). See Section \ref{sec:post-processing} for details of the \texttt{spm} \R\ package for extracting \SPM\ output.

\subsection{Getting help}

\SPM\ is distributed as unsupported software. The authors do not, as a rule, provide help for users of \SPM. However, we may be able to offer limited assistance, and we would appreciate being notified of any problems or errors in \SPM. See Section \ref{sec:reporting-errors} for how to report errors to the \authors. Further information about \SPM\ can be obtained by contacting the \authors.

\subsection{Technical details}

\SPM\ is compiled on Linux using \href{http://gcc.gnu.org}{\texttt{gcc}}, the C/C++ compiler developed by the \href{http://gcc.gnu.org}{GNU Project}. The 32-bit Linux version has been compiled using \texttt{gcc} version 4.1.2 20070115 (prerelease) (\href{http://www.opensuse.org/}{SUSE Linux}), the 64-bit Linux version uses \href{http://gcc.gnu.org}{\texttt{gcc}} version 4.1.0 (\href{http://www.opensuse.org/}{SUSE Linux}). Note that \SPM\ is not supported for Linux kernel versions prior to 2.6. The \href{http://www.microsoft.com}{Microsoft Windows} version is compiled using \href{http://www.mingw.org}{Mingw32} \href{http://gcc.gnu.org}{\texttt{gcc}} 3.4.5, and should run on most 32-bit WindowsXP and Windows Vista systems. There are no plans to port \SPM\ to Microsoft Windows 64-bit platforms. 

\SPM\ uses two minimisers, \textemdash\ the first is closely based on the main algorithm of \cite{779}, and which which uses finite difference gradients, and the second is an implementation of the differential evolution solver \citep{1442}, and based on code by \href{mailto:<godwin@pushcorp.com>}{Lester E. Godwin} of \href{http://www.pushcorp.com}{PushCorp, Inc.} The random number generator used by \SPM\ uses an implementation of the Mersenne twister random number generator \citep{796}. This, the command line functionality, matrix operations, and a number of other functions use the \href{http://www.boost.org/}{BOOST} C++ library (Version 1.38.0).

Note that the output from \SPM\ may differ slightly on the different platforms due to different precision arithmetic or other platform dependent implementation issues. The source code for \SPM\ is available on request.


